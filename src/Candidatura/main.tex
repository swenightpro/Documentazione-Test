\documentclass[12pt, a4paper]{article}
\usepackage[utf8]{inputenc}
\usepackage[italian]{babel}
\usepackage{amsmath}
\usepackage{amssymb}
\usepackage{graphicx}
\usepackage{currfile}
\graphicspath{{src/immagini/}{\currfiledir contenuti/}{\currfiledir contenuti/immagini/}}

\usepackage{url}
\usepackage{longtable} % Per tabelle su più pagine

% Impostazioni della pagina (margini, ecc.)
\usepackage[top=2.5cm, bottom=2.5cm, left=3cm, right=3cm]{geometry}

% Intestazione/Piè di pagina (opzionale, ma utile per numero di pagina e logo)
\usepackage{fancyhdr}
\pagestyle{fancy}
\fancyhead{} % Svuota l'intestazione
\fancyfoot[C]{\thepage} % Numero di pagina centrato nel piè di pagina
\renewcommand{\headrulewidth}{0pt} % Rimuove la linea nell'intestazione
\renewcommand{\footrulewidth}{0.4pt} % Linea sottile nel piè di pagina

\title{} % Lasciato vuoto per un layout più da lettera/documento tecnico
\author{} % Lasciato vuoto

\begin{document}

% Inizio della personalizzazione con il logo NightPRO

\begin{figure}[h]
    \centering
    % Inserisci qui l'immagine del logo NightPRO. 
    % Assicurati che il file sia nella stessa cartella e usa il comando corretto.
    % Esempio: \includegraphics[width=0.4\textwidth]{Chat-GPT-Image-14-ott-2025-18-15-19.jpg}
    \includegraphics[width=0.4\textwidth]{logo.jpg}
\end{figure}

\begin{center}
    \textbf{\Large NightPRO} \\
    \textit{INNOVATE. OPTIMIZE. SECURE.} \\
    \vspace{0.5cm}
    \rule{\linewidth}{0.8pt} % Linea di separazione
    \vspace{0.5cm}
    
    \textbf{Università degli Studi di Padova} \\
    Laurea: Informatica \\
    Corso: [Nome del tuo Corso/Laboratorio] \\ % Esempio: "Ingegneria del Software"
    Anno Accademico: 2025/2026 \\
    \vspace{0.5cm}
    
    \textbf{Documento:} Lettera di Presentazione [o il nome appropriato] \\
    \textbf{Email Gruppo:} [La tua email di gruppo, e.g., info@nightpro.it] \\
\end{center}

\vspace{1cm}

% Tabella delle Informazioni sul Documento (Ispirata alla Page 1 del documento SWEg Labs)
\begin{longtable}{|p{2.5cm}|p{8cm}|}
    \hline
    \textbf{Voce} & \textbf{Dettaglio} \\
    \hline
    \endhead % Ripetizione dell'intestazione per pagine lunghe
    
    \hline
    \textbf{Versione} & 1.0.0 \\
    \hline
    \textbf{Stato} & Bozze/Approvato \\
    \hline
    \textbf{Redazione} & Davide Biasuzzi, [Nome altro membro 1], [Nome altro membro 2] \\
    \hline
    \textbf{Verifica} & [Nome del Verificatore] \\
    \hline
    \textbf{Approvazione} & Tutto il gruppo NightPRO \\
    \hline
    \textbf{Proprietario} & NightPRO \\
    \hline
    \textbf{Uso} & Esterno/Interno \\
    \hline
    \textbf{Destinatari} & [Nome Destinatario/Professore], [Nome Destinatario 2] \\
    \hline
\end{longtable}

\newpage

% Registro delle Modifiche (Ispirato alla Page 2 del documento SWEg Labs)
\section*{Registro delle Modifiche}
\begin{longtable}{|p{1.5cm}|p{2cm}|p{3cm}|p{6cm}|}
    \hline
    \textbf{Vers.} & \textbf{Data} & \textbf{Autore} & \textbf{Descrizione} \\
    \hline
    \endhead
    
    \hline
    1.0.0 & 15-10-2025 & Tutto il gruppo & Approvazione del documento base \\
    \hline
    0.1.0 & 15-10-2025 & Davide Biasuzzi & Creazione della struttura iniziale in LaTeX \\
    \hline
    % Aggiungi altre modifiche qui...
\end{longtable}

\newpage

% Corpo della Lettera (Ispirata alla Page 3 del documento SWEg Labs)
\section*{Lettera di Presentazione}

\noindent Egregio/a [Titolo e Cognome del 1° Destinatario], \\
\noindent Egregio/a [Titolo e Cognome del 2° Destinatario],

\vspace{0.5cm}

Il gruppo \textbf{NightPRO} desidera comunicarvi la sua intenzione a candidarsi ed impegnarsi nella realizzazione del progetto \textbf{[Nome del Vostro Progetto]} (o capitolato \textbf{[Numero/Nome]}).

Siamo un team composto da studenti del corso di Laurea in Informatica dell'Università di Padova, uniti dall'interesse per l'innovazione tecnologica e l'ottimizzazione dei processi, competenze che il nostro capogruppo, \textbf{Davide Biasuzzi} (matricola 2111000 - come da immagine caricata), intende sfruttare anche per l'apertura di una PIVA nel prossimo futuro.

% Inserisci qui il link alla repository (se ne avete una)
La documentazione prodotta dal team relativa al progetto è consultabile alla repository seguente:
\url{https://github.com/NightPRO/[Nome della Vostra Repository]}

All'interno della quale è possibile trovare:
\begin{itemize}
    \item Lettera di presentazione
    \item Valutazioni dei vari capitolati/proposte
    \item Preventivo dei costi e impegno orario totale
    \item Verbali interni/esterni
    % Aggiungi altri elementi...
\end{itemize}

% Qui puoi aggiungere la sezione "Motivazione della Scelta" e "Resoconto degli Incontri"
% seguendo la struttura del documento originale.

\newpage
\section*{I Componenti del Gruppo}

\noindent Di seguito i componenti del gruppo \textbf{NightPRO}:
% Ho inserito il tuo nome e matricola (2111000) dall'immagine che hai caricato,
% insieme agli altri nomi presenti nella stessa riga di tabella.
\begin{longtable}{|l|l|l|}
    \hline
    \textbf{Nome} & \textbf{Cognome} & \textbf{Matricola} \\
    \hline
    \endhead
    
    \hline
    Davide & Biasuzzi & 2111000 \\
    \hline
    Leonardo & Bilato & 2071084 \\
    \hline
    Francesco & Zanella & 2116442 \\
    \hline
    Mihaela-Mariana & Romascu & 2079726 \\
    \hline
    Michele & Ogniben & 2042325 \\
    \hline
    Samuele & Perozzo & 2110989 \\
    \hline
    Giovanni & Ponso & 2000558 \\
    \hline
    % Aggiungi altri membri se necessario
\end{longtable}

\vspace{1cm}

\noindent Cordiali saluti,

\vspace{1cm}
\noindent \textbf{Il gruppo NightPRO}

\end{document}