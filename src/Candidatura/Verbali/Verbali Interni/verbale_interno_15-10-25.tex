\documentclass[a4paper, 11pt, oneside]{scrartcl} % Classe KOMA-Script

% --- Pacchetti Fondamentali ---
\usepackage[utf8]{inputenc}     % Codifica UTF-8
\usepackage[T1]{fontenc}        % Font encoding moderno
\usepackage[italian]{babel}     % Lingua italiana
\usepackage{lmodern}            % Font "Latin Modern"

% --- Grafica e Layout ---
\usepackage{graphicx}           % Per includere immagini
\usepackage{currfile}
\graphicspath{{src/immagini/}{\currfiledir contenuti/}{\currfiledir contenuti/immagini/}}

\usepackage[a4paper, top=2.5cm, bottom=3cm, left=2.5cm, right=2.5cm]{geometry} % Margini
\usepackage{fancyhdr}           % Per header e footer personalizzati
\usepackage{microtype}          % Migliora la tipografia
\usepackage[svgnames]{xcolor}   % Colori

% --- Utility ---
\usepackage{booktabs}           % Tabelle più professionali
\usepackage{enumitem}           % Per personalizzare liste
\usepackage{hyperref}           % Rende i link cliccabili
\hypersetup{
    colorlinks=true,
    linkcolor=DarkBlue,
    filecolor=DarkBlue,      
    urlcolor=DarkBlue,
    citecolor=DarkBlue,
    pdftitle={Documento Progetto - NightPRO},
    pdfauthor={Gruppo NightPRO},
}

% ===================================================================
%  IMPOSTAZIONE HEADER E FOOTER
% ===================================================================
\pagestyle{fancy}
\fancyhf{} % Pulisce tutti i campi
\fancyhead[L]{NightPRO - Progetto Ingegneria del Software}
\fancyhead[R]{Anno Accademico 2025/2026}
\fancyfoot[C]{\thepage} % Numero di pagina al centro in basso
\renewcommand{\headrulewidth}{0.4pt} % Linea sottile sotto l'header
\renewcommand{\footrulewidth}{0pt}

% ===================================================================
%  INIZIO DEL DOCUMENTO
% ===================================================================
\begin{document}

% -------------------------------------------------------------------
%  SEZIONE: intestazione_titolo.tex
% -------------------------------------------------------------------
\thispagestyle{empty}
\begin{titlepage}
    \centering
    
\begin{figure}
    \centering
    \includegraphics[width=0.4\textwidth]{logo.jpg}
\end{figure}

    \vfill
    
    {\small UNIVERSITÀ DEGLI STUDI DI PADOVA \par}
    {\small CORSO DI LAUREA IN INFORMATICA (L-31) \par}
    \vspace{0.5cm}
    {\large Corso di Ingegneria del Software \par}
    {\small Anno Accademico 2025/2026 \par}


    
    \vfill
    
    {\Huge \bfseries Verbale di Riunione \par}
    
    \vspace{1cm}
    
    % Inserisci qui il titolo specifico del documento
    {\Large \itshape Verbale Interno del 15 ottobre 2025 \par} 
    
    \vfill
    
    {\Large \bfseries Gruppo: NightPRO \par}
    \vspace{0.5cm}
    {\large \href{mailto:swe.nightpro@gmail.com}{swe.nightpro@gmail.com} \par}
    
    \vfill
    
    % Inserisci qui la data di redazione del documento
    {\large Data: 15 ottobre 2025 \par}

\end{titlepage}

% -------------------------------------------------------------------
%  SEZIONE: indice.tex
% -------------------------------------------------------------------
\newpage
\tableofcontents % Genera l'indice
\pagestyle{fancy} % Riattiva lo stile di pagina da qui in poi

% -------------------------------------------------------------------
%  SEZIONE: informazioni.tex
% -------------------------------------------------------------------
\newpage
\section{Informazioni Generali}

\subsection{Componenti del Gruppo}
Elenco dei membri del gruppo di lavoro NightPRO.

\begin{table}[h!]
\centering
\begin{tabular}{@{}llc@{}}
\toprule
\textbf{Cognome} & \textbf{Nome} & \textbf{Matricola} \\
\midrule
Biasuzzi & Davide & 2111000 \\
Bilato & Leonardo & 2071084 \\
Zanella & Francesco & 2116442 \\
Romascu & Mihaela-Mariana & 2079726 \\
Ogniben & Michele & 2042325 \\
Perozzo & Samuele & 2110989 \\
Ponso & Giovanni & 2000558 \\
\bottomrule
\end{tabular}
\caption{Componenti del Gruppo NightPRO.}
\end{table}

\subsection{Dettagli Riunione}
\begin{itemize}
    \item \textbf{Data:} 15 ottobre 2025
    \item \textbf{Ora:} 18:00 - 19:30
    \item \textbf{Luogo:} Google Meet
    \item \textbf{Partecipanti:} Tutti i membri del gruppo hanno partecipato
\end{itemize}


% -------------------------------------------------------------------
%  SEZIONE: odg.tex (Ordine del Giorno)
% -------------------------------------------------------------------
\newpage
\section{Ordine del Giorno (Agenda)}
\begin{itemize}
    \item[1.] Discussione generale su tutti i capitolati
    \item[2.] Approfondimento su capitolato C6
    \item[3.] Approfondimento su capitolato C8
    \item[4.] Approfondimento su capitolato C9
    \item[5.] Contatti email da effettuare
\end{itemize}

% -------------------------------------------------------------------
%  SEZIONE: diario.tex (Diario della riunione)
% -------------------------------------------------------------------
\newpage
\section{Diario della Riunione}

\subsection{Discussione generale su tutti i capitolati:}

Si è proceduto all'analisi condivisa del file Google contenente le preferenze e le votazioni sui capitolati, preparato il giorno precedente. Dopo una discussione generale su tutte le proposte, il gruppo ha identificato tre capitolati di maggiore interesse sui quali concentrare l'approfondimento:
\begin{itemize}
    \item C6 (Second Brain)
    \item C8 (Smart Order)
    \item C9 (View4Life)
\end{itemize}


\subsection{Approfondimento su capitolato C6:}
È stato analizzato in dettaglio il capitolato C6. Sono emersi diversi dubbi e punti chiave da chiarire con il proponente (Dott. Piccoli) al fine di definire correttamente l'ambito del Proof of Concept (PoC) e dell'MVP (Minimum Viable Product). 
La discussione si è focalizzata su:
\begin{itemize}
    \item Architettura Server: Si è discusso sul carattere "opzionale" della parte server. Si è convenuto che per le funzionalità LLM avanzate (come il collegamento tra note) sia fondamentale. È necessario chiedere se il PoC debba già includere la predisposizione per un database server-side.
    \item Tecnologie Web App: È stata notata la richiesta di un editor MarkDown in pagina HTML. Si è deciso di chiedere a Zucchetti se esistano preferenze tecnologiche (es. specifici framework UI) oltre a quelle indicate.
    \item Versionamento delle Note: Il capitolato menziona il salvataggio in file di testo. Si è discusso se sia sufficiente un semplice salvataggio (sovrascrittura) o se sia atteso un meccanismo di versioning dei contenuti.
\end{itemize}

\subsection{Approfondimento su capitolato C8 (Smart Order):}
Analogamente, si è approfondito il capitolato C8. L'analisi dell'architettura proposta ha sollevato diversi quesiti tecnici fondamentali da porre al proponente (Dott. Carlesso) per definire i confini del progetto. 
I punti di discussione principali sono stati:
\begin{itemize}
    \item Modalità di Input: Chiarire quale sia il numero minimo di modalità (audio, testo, immagine, ecc.) richieste per il prodotto finale, oltre alle 1-2 suggerite per il PoC.
    \item Confini del Sistema (Layer 1): Definire se la progettazione dei connettori per i dati eterogenei rientri nei compiti del gruppo, o se si debba assumere che l'infrastruttura dati sia già fornita.
    \item Livello di Automazione (Layer 6): Comprendere se il layer di validazione e arricchimento dati debba tendere a un'automazione completa o se prevedere un intervento di supervisione umana (human-in-the-loop).
    \item Gestione Retraining (Layer 8): Approfondire se il "retraining periodico" menzionato debba essere implementato come un ciclo automatico basato sul feedback o come un'operazione supervisionata.
    \item Stack Tecnologico: Verificare se Ergon abbia uno stack tecnologico di riferimento interno o preferenze specifiche (es. .NET Blazor, React) per l'interfaccia.
\end{itemize}
\subsection{Approfondimento su capitolato C9 (View4Life):}
Per il capitolato C9, la discussione si è concentrata principalmente su aspetti organizzativi e gestionali, ritenuti critici per la fattibilità del progetto.
\begin{itemize}
    \item Gestione Comunicazioni: È emersa una preoccupazione basata su feedback (riportati da studenti degli anni precedenti) circa possibili lunghi tempi di risposta da parte dei proponenti. Si è deciso di chiedere preventivamente e con trasparenza se per l'anno corrente siano state previste modalità o tempistiche specifiche per la gestione delle domande dei gruppi.
\end{itemize}
\subsection{Contatti email da effettuare:}
Al termine della discussione, si è deciso di contattare via email tutti e tre i proponenti per sottoporre i dubbi emersi e ottenere i chiarimenti necessari prima di una scelta definitiva.
Sono stati assegnati i seguenti incarichi per la stesura e l'invio delle comunicazioni:

\begin{itemize}
    \item C6 (Second Brain): Davide Biasuzzi
    \item C8 (Smart Order): Giovanni Ponso
    \item C9 (View4Life): Leonardo Bilato
\end{itemize}
% -------------------------------------------------------------------
%  SEZIONE: decisioni.tex (Decisioni prese)
% -------------------------------------------------------------------
\newpage
\section{Decisioni Prese}

\begin{enumerate}
    \item Scrivere mail ai capitolati d'interesse con lo scopo di approfondire e se necessario fissare un colloquio con l'azienda
\end{enumerate}

% -------------------------------------------------------------------
%  SEZIONE: todo.tex (Attività da svolgere)
% -------------------------------------------------------------------
\newpage
\section{Attività da Svolgere (To-Do)}

\begin{table}[h!]
\centering
\begin{tabular}{@{}lll@{}}
\toprule
\textbf{Attività} & \textbf{Assegnatario/i} & \textbf{Scadenza} \\
\midrule
Inviare mail per capitolato C6 & Davide Biasuzzi & 16/10/25 \\
Inviare mail per capitolato C8 & Giovanni Ponso & 16/10/25 \\
Inviare mail per capitolato C9 & Francesco Zanella & 16/10/25 \\
\bottomrule
\end{tabular}
\caption{Riepilogo task assegnati.}
\end{table}

\end{document}