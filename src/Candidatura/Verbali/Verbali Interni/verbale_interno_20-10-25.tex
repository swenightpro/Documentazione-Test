% ===================================================================
%  TEMPLATE DOCUMENTO PROGETTO (File Unico)
%  Gruppo: NightPRO
%  Progetto: Ingegneria del Software 2025/2026
%
%  Basato sulla struttura:
%  - configurazione.tex
%  - intestazione_titolo.tex
%  - indice.tex
%  - informazioni.tex
%  - odg.tex
%  - diario.tex
%  - decisioni.tex
%  - todo.tex
% ===================================================================

% -------------------------------------------------------------------
%  SEZIONE: configurazione.tex
% -------------------------------------------------------------------
\documentclass[a4paper, 11pt, oneside]{scrartcl} % Classe KOMA-Script

% --- Pacchetti Fondamentali ---
\usepackage[utf8]{inputenc}     % Codifica UTF-8
\usepackage[T1]{fontenc}        % Font encoding moderno
\usepackage[italian]{babel}     % Lingua italiana
\usepackage{lmodern}            % Font "Latin Modern"

% --- Grafica e Layout ---
\usepackage{graphicx}           % Per includere immagini
\usepackage{currfile}
\graphicspath{{src/immagini/}{\currfiledir contenuti/}{\currfiledir contenuti/immagini/}}

\usepackage[a4paper, top=2.5cm, bottom=3cm, left=2.5cm, right=2.5cm]{geometry} % Margini
\usepackage{fancyhdr}           % Per header e footer personalizzati
\usepackage{microtype}          % Migliora la tipografia
\usepackage[svgnames]{xcolor}   % Colori

% --- Utility ---
\usepackage{booktabs}           % Tabelle più professionali
\usepackage{enumitem}           % Per personalizzare liste
\usepackage{hyperref}           % Rende i link cliccabili
\hypersetup{
    colorlinks=true,
    linkcolor=DarkBlue,
    filecolor=DarkBlue,      
    urlcolor=DarkBlue,
    citecolor=DarkBlue,
    pdftitle={Documento Progetto - NightPRO},
    pdfauthor={Gruppo NightPRO},
}

% ===================================================================
%  IMPOSTAZIONE HEADER E FOOTER
% ===================================================================
\pagestyle{fancy}
\fancyhf{} % Pulisce tutti i campi
\fancyhead[L]{NightPRO - Progetto Ingegneria del Software}
\fancyhead[R]{Anno Accademico 2025/2026}
\fancyfoot[C]{\thepage} % Numero di pagina al centro in basso
\renewcommand{\headrulewidth}{0.4pt} % Linea sottile sotto l'header
\renewcommand{\footrulewidth}{0pt}

% ===================================================================
%  INIZIO DEL DOCUMENTO
% ===================================================================
\begin{document}

% -------------------------------------------------------------------
%  SEZIONE: intestazione_titolo.tex
% -------------------------------------------------------------------
\thispagestyle{empty}
\begin{titlepage}
    \centering
    
\begin{figure}
    \centering
    \includegraphics[width=0.4\textwidth]{logo.jpg}
\end{figure}

    \vfill
    
    {\small UNIVERSITÀ DEGLI STUDI DI PADOVA \par}
    {\small CORSO DI LAUREA IN INFORMATICA (L-31) \par}
    \vspace{0.5cm}
    {\large Corso di Ingegneria del Software \par}
    {\small Anno Accademico 2025/2026 \par}


    
    \vfill
    
    {\Huge \bfseries Verbale di Riunione \par}
    
    \vspace{1cm}
    
    % Inserisci qui il titolo specifico del documento
    {\Large \itshape Verbale Interno del 20 ottobre 2025 \par} 
    
    \vfill
    
    {\Large \bfseries Gruppo: NightPRO \par}
    \vspace{0.5cm}
    {\large \href{mailto:swe.nightpro@gmail.com}{swe.nightpro@gmail.com} \par}
    
    \vfill
    
    % Inserisci qui la data di redazione del documento
    {\large Data: 20 ottobre 2025 \par}

\end{titlepage}

% -------------------------------------------------------------------
%  SEZIONE: indice.tex
% -------------------------------------------------------------------
\newpage
\tableofcontents % Genera l'indice
\pagestyle{fancy} % Riattiva lo stile di pagina da qui in poi

% -------------------------------------------------------------------
%  SEZIONE: informazioni.tex
% -------------------------------------------------------------------
\newpage
\section{Informazioni Generali}

\subsection{Componenti del Gruppo}
Elenco dei membri del gruppo di lavoro NightPRO.

\begin{table}[h!]
\centering
\begin{tabular}{@{}llc@{}}
\toprule
\textbf{Cognome} & \textbf{Nome} & \textbf{Matricola} \\
\midrule
Biasuzzi & Davide & 2111000 \\
Bilato & Leonardo & 2071084 \\
Zanella & Francesco & 2116442 \\
Romascu & Mihaela-Mariana & 2079726 \\
Ogniben & Michele & 2042325 \\
Perozzo & Samuele & 2110989 \\
Ponso & Giovanni & 2000558 \\
\bottomrule
\end{tabular}
\caption{Componenti del Gruppo NightPRO.}
\end{table}

\subsection{Dettagli Riunione}
\begin{itemize}
    \item \textbf{Data:} 20 ottobre 2025
    \item \textbf{Ora:} 17:30 - 19:30
    \item \textbf{Luogo:} Google Meet
    \item \textbf{Partecipanti:} Biasuzzi Davide, Zanella Francesco, Ponso Giovanni, Bilato Leonardo
    \item  \textbf{Redatto da: } Francesco Zanella
    \item  \textbf{Verificato da:} Davide Biasuzzi
    \item \textbf{Versione: } 1.0
\end{itemize}


% -------------------------------------------------------------------
%  SEZIONE: odg.tex (Ordine del Giorno)
% -------------------------------------------------------------------
\newpage
\section{Ordine del Giorno (Agenda)}
\begin{itemize}
    \item[1.] Funzionamento Github
    \item[2.] Compilazione documenti Github
    \item[3.] Decisione documenti da iniziare a produrre
\end{itemize}

% -------------------------------------------------------------------
%  SEZIONE: diario.tex (Diario della riunione)
% -------------------------------------------------------------------
\newpage
\section{Diario della Riunione}

\subsection{Funzionamento Github:}
Il gruppo ha avviato la riunione con una breve panoramica su GitHub, volta a chiarire il suo funzionamento di base.
È emerso che uno dei membri aveva recentemente approfondito l’uso della piattaforma e ha condiviso le nozioni principali con il resto del gruppo, illustrando i concetti di repository, commit, branch e pull request.
Un secondo membro aveva già utilizzato GitHub in passato, sebbene in modo limitato, e ha contribuito con esempi pratici su come gestire file e versioni.
La discussione si è concentrata anche su come orientarsi nell’interfaccia e come creare e caricare vari tipi di file.
L’obiettivo della sezione era garantire che tutti i membri acquisissero un livello minimo comune di competenza per poter collaborare in modo efficace sulla piattaforma.

\subsection{Compilazione documenti Github:}
Successivamente, il gruppo ha analizzato come procedere alla redazione e compilazione dei documenti direttamente su GitHub.
È stato chiarito il funzionamento della modifica dei file in formato Markdown e la possibilità di lavorare in modo collaborativo, evitando sovrapposizioni grazie all’uso dei branch e delle commit message chiare e coerenti.
Si è inoltre discusso della struttura da mantenere all’interno del repository per rendere ordinata la documentazione (cartelle dedicate ai verbali, ai template, ecc.).
Il gruppo ha concordato di utilizzare una nomenclatura coerente per i file e di mantenere una cronologia trasparente delle modifiche, in modo da facilitare eventuali revisioni future.
\subsection{Decisione documenti da iniziare a produrre:}
Nella parte finale della riunione si è deciso di iniziare la produzione dei seguenti documenti:
\begin{itemize}
    \item Diario di bordo, per tenere traccia delle attività svolte e delle decisioni prese durante le varie fasi del progetto.
    \item Verbali delle riunioni, per documentare gli incontri in modo formale e consultabile da tutti i membri.
    \item  Documento di candidatura, necessario per presentare ufficialmente il progetto e il gruppo di lavoro.
\end{itemize}

Il gruppo ha concordato di partire da modelli condivisi su GitHub, in modo da mantenere uniformità nella struttura dei documenti (template).

% -------------------------------------------------------------------
%  SEZIONE: decisioni.tex (Decisioni prese)
% -------------------------------------------------------------------
\newpage
\section{Decisioni Prese}

\begin{enumerate}
    \item Gestione file all'interno di Github
    \item Priorità nella stesura della documentazione necessaria
\end{enumerate}

% -------------------------------------------------------------------
%  SEZIONE: todo.tex (Attività da svolgere)
% -------------------------------------------------------------------
\newpage
\section{Attività da Svolgere (To-Do)}

\begin{table}[h!]
\centering
\begin{tabular}{@{}lll@{}}
\toprule
\textbf{Attività} & \textbf{Assegnatario/i} & \textbf{Scadenza} \\
\midrule
Ottimizzazione e divisione su Github & Leonardo Bilato & 21/10/25 \\
Stesura template& Davide Biasuzzi & 21/10/25 \\
stesura prime relazioni& Francesco Zanella & 21/10/25 \\
\bottomrule
\end{tabular}
\caption{Riepilogo task assegnati.}
\end{table}

\end{document}