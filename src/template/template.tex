% ===================================================================
%  TEMPLATE DOCUMENTO PROGETTO (File Unico)
%  Gruppo: NightPRO
%  Progetto: Ingegneria del Software 2025/2026
% ===================================================================

% -------------------------------------------------------------------
%  SEZIONE: configurazione
% -------------------------------------------------------------------
\documentclass[a4paper, 11pt, oneside]{scrartcl} % Classe KOMA-Script

% --- Pacchetti Fondamentali ---
\usepackage[utf8]{inputenc}     % Codifica UTF-8
\usepackage[T1]{fontenc}        % Font encoding moderno
\usepackage[italian]{babel}     % Lingua italiana
\usepackage{lmodern}            % Font "Latin Modern"

% --- Grafica e Layout ---
\usepackage{graphicx}           % Per includere immagini
\usepackage{currfile}
\graphicspath{{src/immagini/}{\currfiledir contenuti/}{\currfiledir contenuti/immagini/}}

\usepackage[a4paper, top=2.5cm, bottom=3cm, left=2.5cm, right=2.5cm]{geometry} % Margini
\usepackage{fancyhdr}           % Per header e footer
\usepackage{microtype}          % Migliora la tipografia
\usepackage[svgnames]{xcolor}   % Colori

% --- Utility ---
\usepackage{booktabs}           % Tabelle più professionali
\usepackage{enumitem}           % Per personalizzare liste
\usepackage{hyperref}           % Rende i link cliccabili
\hypersetup{
    colorlinks=true,
    linkcolor=DarkBlue,
    filecolor=DarkBlue,      
    urlcolor=DarkBlue,
    citecolor=DarkBlue,
    pdftitle={Documento Progetto - NightPRO},
    pdfauthor={Gruppo NightPRO},
}

% ===================================================================
%  IMPOSTAZIONE HEADER E FOOTER
% ===================================================================
\pagestyle{fancy}
\fancyhf{} % Pulisce tutti i campi
\fancyhead[L]{NightPRO - Progetto Ingegneria del Software}
\fancyhead[R]{Anno Accademico 2025/2026}
\fancyfoot[C]{\thepage} % Numero di pagina al centro in basso
\renewcommand{\headrulewidth}{0.4pt} % Linea sottile sotto l'header
\renewcommand{\footrulewidth}{0pt}

% ===================================================================
%  INIZIO DEL DOCUMENTO
% ===================================================================
\begin{document}

% -------------------------------------------------------------------
%  SEZIONE: intestazione_titolo
% -------------------------------------------------------------------
\thispagestyle{empty}
\begin{titlepage}
    \centering
    
\begin{figure}
    \centering
    \includegraphics[width=0.4\textwidth]{logo.jpg}
\end{figure}

    \vfill
    
    {\small UNIVERSITÀ DEGLI STUDI DI PADOVA \par}
    {\small CORSO DI LAUREA IN INFORMATICA (L-31) \par}
    \vspace{0.5cm}
    {\large Corso di Ingegneria del Software \par}
    {\small Anno Accademico 2025/2026 \par}


    
    \vfill
    
    {\Huge \bfseries Documentazione di Progetto \par}
    
    \vspace{1cm}
    
    % Inserisci qui il titolo specifico del documento
    {\Large \itshape (Titolo Specifico del Documento) \par} 
    
    \vfill
    
    {\Large \bfseries Gruppo: NightPRO \par}
    \vspace{0.5cm}
    {\large \href{mailto:swe.nightpro@gmail.com}{swe.nightpro@gmail.com} \par}
    
    \vfill
    
    % Inserisci qui la data di redazione del documento
    {\large Data: \today \par}

\end{titlepage}

% -------------------------------------------------------------------
%  SEZIONE: indice
% -------------------------------------------------------------------
\newpage
\tableofcontents % Genera l'indice
\pagestyle{fancy} % Riattiva lo stile di pagina da qui in poi

% -------------------------------------------------------------------
%  SEZIONE: informazioni
% -------------------------------------------------------------------
\newpage
\section{Informazioni Generali}

\subsection{Componenti del Gruppo}
Elenco dei membri del gruppo di lavoro NightPRO.

\begin{table}[h!]
\centering
\begin{tabular}{@{}llc@{}}
\toprule
\textbf{Cognome} & \textbf{Nome} & \textbf{Matricola} \\
\midrule
Biasuzzi & Davide & 2111000 \\
Bilato & Leonardo & 2071084 \\
Zanella & Francesco & 2116442 \\
Romascu & Mihaela-Mariana & 2079726 \\
Ogniben & Michele & 2042325 \\
Perozzo & Samuele & 2110989 \\
Ponso & Giovanni & 2000558 \\
\bottomrule
\end{tabular}
\caption{Componenti del Gruppo NightPRO.}
\end{table}

\subsection{Architettura dei Repository}
Il progetto è articolato su due repository distinti:
\begin{itemize}
    \item \textbf{Repository Documentazione:} \url{https://github.com/swenightpro/Documentazione}
    \item \textbf{Repository Prodotto Software:} \url{https://github.com/swenightpro/nightpro}
\end{itemize}

\subsection{Dettagli Riunione (se applicabile)}
\begin{itemize}
    \item \textbf{Data:} (Inserire data riunione)
    \item \textbf{Luogo:} (Inserire luogo)
    \item \textbf{Partecipanti:} (Inserire presenti/assenti)
\end{itemize}


% -------------------------------------------------------------------
%  SEZIONE: odg (Ordine del Giorno)
% -------------------------------------------------------------------
\newpage
\section{Ordine del Giorno (Agenda)}
\emph{(Questa sezione descrive i punti da discutere)}

\begin{itemize}
    \item[1.] (Punto 1 dell'agenda)
    \item[2.] (Punto 2 dell'agenda)
    \item[3.] ...
    \item[4.] Varie ed eventuali
\end{itemize}

% -------------------------------------------------------------------
%  SEZIONE: diario (Diario della riunione)
% -------------------------------------------------------------------
\newpage
\section{Diario della Riunione}
\emph{(Questa sezione trascrive la discussione avvenuta)}

\subsection{Discussione Punto 1: ...}
(Trascrizione della discussione...)

\subsection{Discussione Punto 2: ...}
(Trascrizione della discussione...)

\subsection{Discussione Varie ed eventuali}
(Trascrizione della discussione...)

% -------------------------------------------------------------------
%  SEZIONE: decisioni (Decisioni prese)
% -------------------------------------------------------------------
\newpage
\section{Decisioni Prese}
\emph{(Questa sezione elenca le decisioni chiave concordate)}

\begin{enumerate}
    \item (Decisione 1)
    \item (Decisione 2)
    \item ...
\end{enumerate}

% -------------------------------------------------------------------
%  SEZIONE: todo.tex (Attività da svolgere)
% -------------------------------------------------------------------
\newpage
\section{Attività da Svolgere (To-Do)}
\emph{(Questa sezione elenca i task assegnati)}

\begin{table}[h!]
\centering
\begin{tabular}{@{}lll@{}}
\toprule
\textbf{Attività} & \textbf{Assegnatario/i} & \textbf{Scadenza} \\
\midrule
(Task 1) & (Nome) & (gg/mm/aaaa) \\
(Task 2) & (Nome) & (gg/mm/aaaa) \\
(Task 3) & (Nome) & (gg/mm/aaaa) \\
... & ... & ... \\
\bottomrule
\end{tabular}
\caption{Riepilogo task assegnati.}
\end{table}

\end{document}